% -*-LaTeX-*-
% $Id: abstract.tex 70 2007-01-30 21:59:16Z nicolosi $

\begin{abstract}
Memory intensive applications such as {\emph{in-memory key-value stores, web server accelerators, etc.}} require large amounts of memory. However, these applications have a highly skewed memory access footprint such that a small subset of objects comprise most of the memory accesses. In this paper, we present the design and implementation of Tracer: a tool that allows developers to trace memory accesses at object level granularity. The design philosophy of Tracer is to assist such applications in identifying {\emph{entities}} (such as C level {\emph{structures}}) which get heavily accessed. Tracer accomplishes this with negligible intrusion into the application code, allowing for easy integration into existing projects. We evaluated a prototype implementation of Tracer and observed reasonable overheads of $28\%$ and $40\%$ over non-modified C code for creation and read heavy workloads. As compared to a page protection mechanism, Tracer improves the average performance by $12\%$ and $6.28\%$ for read and update heavy workloads. 
\end{abstract}

