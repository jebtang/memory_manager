\section{Motivation}
\label{sec:motivation}
\paragraph{Memory Augmentation}
Modern web-scale applications (web accelerators, proxy servers, in-memory key-value stores) tend to rely heavily on data (indices, key-value pairs) cached in DRAM as their primary storage for faster access. Improving the utilization of the available DRAM can reduce the latencies incurred from accessing HDDs and SSDs. Determining which objects are most frequently accessed would allow developers to pack those objects in DRAM and minimize disk access latencies. Additionally, alternative memory solutions such as Storage Class Memories (SCM) could augment the DRAM by being available either as disk caches or slower DRAMs. 

\paragraph{Memory Tiering}
Efficient placement of data at different levels within the memory hierarchy is referred to as {\emph{memory tiering}}. One of the key requirements of {\emph{memory tiering}} is the {\emph{transparent}} movement of application data between different layers of the hierarchy. The adaptability of such {\emph{hybrid memory hierarchies}} depends on the unobtrusiveness of software technologies supporting these memories. Tiering data in memory hierarchies requires an understanding of memory access patterns that is unlikely to be clear to the programmer from the application's design alone. 

\paragraph{Memory Tracing}
{\emph{Memory tracing}} techniques monitor application data at a fine granularity and provide insights into better memory management techniques. Existing{\emph{Memory tracing}} techniques such as {\emph{memcheck, ptrace, data flow tracking mechanisms (DFTs), binary rewriting mechanisms \cite{newsome2005dynamic, payerlightweight}} introduce high overheads, and so finding new techniques with smaller overheads can help improve the utilization of these applications. 
