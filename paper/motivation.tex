\section{Motivation}
\label{sec:motivation}
\paragraph{Memory Augmentation}
Modern web-scale applications (web accelerators, proxy servers, in-memory key-value stores)[4] tend to rely heavily on data (indices, key-value pairs) cached in DRAM as their primary storage for faster access. Improving the utilization of the available DRAM can reduce the dependency on HDDs, SSDs. Additionally, alternate memory solutions such as Storage Class Memories (SCM) could augment the DRAM either by being available as disk caches or slower DRAMs. 

\paragraph{Memory Tiering}
Efficient placement of data at different levels within the memory hierarchy is referred to as {\emph{memory tiering}}. One of the key requirements of {\emph{memory tiering}} would be {\emph{transparent}} movement of application data between the different layers of hierarchy. The adaptability of such {\emph{hybrid memory hierarchies}} would depend on the unobtrusiveness of software technologies which support these memories. Tiering data between memory hierarchies would require an understanding of memory access patterns. 

\paragraph{Memory Tracing}
{\emph{Memory tracing}} techniques monitor application data at finer granularity and provide insights into better memory management techniques. {\emph{Memory tracing}} techniques such as {\emph{memcheck, ptrace, data flow tracking mechanisms (DFTs), binary rewriting mechanisms \cite{taint check, memTrace}} introduce higher overheads. This work explores code instrumentation and object tagging mechanisms in order to build efficient memory monitoring libraries.
