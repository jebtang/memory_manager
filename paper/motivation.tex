\section{Motivation}
\label{sec:motivation}
\paragraph{Memory Augmentation}
Modern web-scale applications (web accelerators, proxy servers, in-memory key-value stores) rely heavily on data (indices, key-value pairs) cached in DRAM as their primary storage. Improvement in the utilization of available DRAM can reduce the latencies incurred from accessing seconadry stores (SSDs, HDDs). Determining which objects are most frequently accessed would allow developers to pack those objects in DRAM and minimize latency due to disk accesses. Additionally, alternative memory solutions such as {\emph{Storage Class Memories}} (such as memristors, phase change memory, etc.) \cite{lam2010storage} would augment the DRAM by being available either as disk caches or slower DRAMs. 

\paragraph{Memory Tiering}
Efficient placement of data at different levels within the memory hierarchy is referred to as {\emph{memory tiering}}. One of the key requirements of {\emph{memory tiering}} is {\emph{transparency}} in the movement of application's data between different layers of memory hierarchy. The adaptability of such {\emph{hybrid memory systems}} would depend on the unobtrusiveness of software technologies supporting them. Tiering data in memory hierarchies requires an understanding of memory access patterns that is unlikely to be clear to the programmer from the application's design alone. 

\paragraph{Memory Tracing}
{\emph{Memory tracing}} techniques monitor application data at a fine granularity and provide insights into better memory management. Existing{\emph{Memory tracing}} techniques such as {\emph{memcheck, ptrace, data flow tracking mechanisms (DFTs), binary translation mechanisms \cite{newsome2005dynamic, payerlightweight}} introduce high overheads, and so finding new techniques with smaller overheads can help improve an application's performance. 
